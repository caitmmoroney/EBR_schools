\documentclass{article}

\usepackage{arxiv}

\usepackage[utf8]{inputenc} % allow utf-8 input
\usepackage[T1]{fontenc}    % use 8-bit T1 fonts
\usepackage{hyperref}       % hyperlinks
\usepackage{url}            % simple URL typesetting
\usepackage{booktabs}       % professional-quality tables
\usepackage{amsfonts}       % blackboard math symbols
\usepackage{nicefrac}       % compact symbols for 1/2, etc.
\usepackage{microtype}      % microtypography
\usepackage{lipsum}
\usepackage{graphicx}

\title{School District Secession in East Baton Rouge}

\author{
    Caitlin Moroney
   \\
    Department of Mathematics \& Statistics \\
    American University \\
  Washington, DC \\
  \texttt{\href{mailto:cm0246b@american.edu}{\nolinkurl{cm0246b@american.edu}}} \\
  }


% Pandoc citation processing

\begin{document}
\maketitle

\def\tightlist{}


\begin{abstract}
Enter the text of your abstract here.
\end{abstract}

\keywords{
    education
   \and
    race
   \and
    schools
   \and
    school district secession
   \and
    school district fragmentation
  }

\hypertarget{proposal}{%
\section{Proposal}\label{proposal}}

I am interested in examining the relationships between school
performance, race, and socioeconomic indicators (including income and
percentage of ``economically disadvantaged''{[}\^{}I use the Louisiana
Department of Education's data on percentages of economically
disadvantaged students. I have yet to find sufficient metadata
explaining how they define ``economically disadvantaged,'' but comparing
current school and school district performance reports with past reports
suggests that this may be the percentage of students on free or reduced
lunch plans.{]} students). Furthermore, I am interested in examining
this data within the context of school district ``secession.''\footnote{See
  CITATION HERE for a thorough explanation of the school district
  ``secession'' phenomenon.} I have chosen to focus on schools within
East Baton Rouge Parish, Louisiana for two reasons: (1) there are stark
differences among school districts with regard to school and school
district performance {[}CITATION HERE{]}, and (2) residents in the
southeast corner of the Parish recently voted to incorporate as the City
of St.~George with the intention of forming a new school district
{[}CITATION HERE{]}. If successful in creating a new school district,
St.~George will be the fourth new district to ``secede'' from the
original East Baton Rouge Parish School District in two decades.

I propose to investigate the following research questions:

\begin{itemize}
\tightlist
\item
  Is there evidence that suggests the clustering of racial groups in
  East Baton Rouge (EBR) Parish?
\item
  Is there evidence of the clustering of income levels in EBR Parish?
\item
  Is there evidence of racial stratification of income?
\item
  Is there evidence to suggest that school performance is linked to the
  racial makeup of the student body, the percentage of economically
  disadvantaged students, and/or the percentage of students fully
  proficient in English?
\end{itemize}

\hypertarget{abstract}{%
\section{Abstract}\label{abstract}}

\label{sec:abstract}

\hypertarget{introduction}{%
\section{Introduction}\label{introduction}}

\hypertarget{methods}{%
\section{Methods}\label{methods}}

\label{sec:methods}

\hypertarget{results}{%
\section{Results}\label{results}}

\label{sec:results}

\hypertarget{discussion}{%
\section{Discussion}\label{discussion}}

\hypertarget{citations}{%
\section{Citations}\label{citations}}

\bibliographystyle{unsrt}
\bibliography{references.bib}


\end{document}
